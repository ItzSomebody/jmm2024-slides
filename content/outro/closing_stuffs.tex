\section{Possible Future Research}

\begin{frame}{Future Research Possibilities?}
    \begin{enumerate}
        \item General $n$-face urban renewal corresponds to QP mutation for the special case $n = 4$. What properties arise when $n \neq 4$?
        \begin{center}
            \begin{tikzpicture}[scale=0.25,baseline=(current bounding box.center)]
                \begin{scope}[every node/.style={fill,circle,inner sep=2pt}]
                    
                    \node (D1) at (2.0,3.8301) {};
                    \node (D3) at (-4.5,0.0) {};
                    \node (D5) at (2.0,-3.8301) {};
                \end{scope}
                \begin{scope}[every node/.style={draw,thick,circle,inner sep=2pt}]
                    
                    \node (D2) at (-2.0,3.8301) {};
                    \node (D4) at (-2.0,-3.8301) {};
                    \node (D6) at (4.5,-0.0) {};
                \end{scope}
                \begin{scope}[every edge/.style={draw,thick}]
                    \path (D1) edge (D2);
                    \path (D2) edge (D3);
                    \path (D3) edge (D4);
                    \path (D4) edge (D5);
                    \path (D5) edge (D6);
                    \path (D6) edge (D1);
                \end{scope}
                \begin{scope}[every node/.style={fill=red,circle,inner sep=2pt}]
                    \node (Q0) at (0,0) {};
                    \node (Q1) at (0.0,4.5) {};
                    \node (Q2) at (-3.8971,2.25) {};
                    \node (Q3) at (-3.8971,-2.25) {};
                    \node (Q4) at (-0.0,-4.5) {};
                    \node (Q5) at (3.8971,-2.25) {};
                    \node (Q6) at (3.8971,2.25) {};  
                \end{scope}
                \begin{scope}[>={Stealth[red,inset=1pt,length=5pt]},
                    every edge/.style={draw=red,thick},
                    every node/.style={fill=white,circle,inner sep=0.7pt}]
                    \path[->] (Q1) edge (Q0);
                    \path[<-] (Q2) edge (Q0);
                    \path[->] (Q3) edge (Q0);
                    \path[<-] (Q4) edge (Q0);
                    \path[->] (Q5) edge (Q0);
                    \path[<-] (Q6) edge (Q0);
                \end{scope}
            \end{tikzpicture}
            \begin{tikzpicture}[baseline=(current bounding box.center)]
                \begin{scope}[>={Stealth[black,inset=0pt,length=7pt]},
                    every edge/.style={draw=black,thick}]
                    
                    \node (A) at (0,0) {};
                    \node (B) at (1.5,0) {};
                    \path[->] (A) edge (B);
                \end{scope}
            \end{tikzpicture}
            \begin{tikzpicture}[scale=0.25,baseline=(current bounding box.center)]
                \begin{scope}[every node/.style={fill,circle,inner sep=2pt}]
                    
                    \node (D1) at (2.5,4.3301) {};
                    \node (D3) at (-5.0,0.0) {};
                    \node (D5) at (2.5,-4.3301) {};
    
                    \node (X2) at (-1.5,2.5981) {};
                    \node (X4) at (-1.5,-2.5981) {};
                    \node (X6) at (3.0,-0.0) {};
                \end{scope}
                \begin{scope}[every node/.style={draw,thick,circle,inner sep=2pt}]
                    
                    \node (D2) at (-2.5,4.3301) {};
                    \node (D4) at (-2.5,-4.3301) {};
                    \node (D6) at (5.0,-0.0) {};
    
                    \node (X1) at (1.5,2.5981) {};
                    \node (X3) at (-3.0,0.0) {};
                    \node (X5) at (1.5,-2.5981) {};
                \end{scope}
                \begin{scope}[every edge/.style={draw,thick}]
                    \path (D1) edge (X1);
                    \path (D2) edge (X2);
                    \path (D3) edge (X3);
                    \path (D4) edge (X4);
                    \path (D5) edge (X5);
                    \path (D6) edge (X6);
    
                    \path (X1) edge (X2);
                    \path (X2) edge (X3);
                    \path (X3) edge (X4);
                    \path (X4) edge (X5);
                    \path (X5) edge (X6);
                    \path (X6) edge (X1);
                \end{scope}
                \begin{scope}[every node/.style={fill=red,circle,inner sep=2pt}]
                    \node (Q0) at (0,0) {};
                    \node (Q1) at (0.0,4.0) {};
                    \node (Q2) at (-3.4641,2.0) {};
                    \node (Q3) at (-3.4641,-2.0) {};
                    \node (Q4) at (-0.0,-4.0) {};
                    \node (Q5) at (3.4641,-2.0) {};
                    \node (Q6) at (3.4641,2.0) {};            
                \end{scope}
                \begin{scope}[>={Stealth[red,inset=0pt,length=5pt]},
                    every edge/.style={draw=red,thick},
                    every node/.style={fill=white,circle,inner sep=0.7pt}]
        
                    \path[->] (Q1) edge[bend right=25] (Q2);
                    \path[<-] (Q2) edge[bend right=25] (Q3);
                    \path[->] (Q3) edge[bend right=25] (Q4);
                    \path[<-] (Q4) edge[bend right=25] (Q5);
                    \path[->] (Q5) edge[bend right=25] (Q6);
                    \path[<-] (Q6) edge[bend right=25] (Q1);
    
                    \path[<-] (Q1) edge (Q0);
                    \path[->] (Q2) edge (Q0);
                    \path[<-] (Q3) edge (Q0);
                    \path[->] (Q4) edge (Q0);
                    \path[<-] (Q5) edge (Q0);
                    \path[->] (Q6) edge (Q0);
                \end{scope}
            \end{tikzpicture}
        \end{center}

        \vfill
        
        \item Formulating general $n$-face urban renewal in terms of a ribbon graph framework. Avoids ``picture-proofs'' and some topology pathologies \cite{bocklandtDimerABC2015}
        
        \vfill
        
        \item Connections to Postnikov diagrams and plabic graphs (and possibly positroids)?
    \end{enumerate}
\end{frame}


\section{Acknowledgements}

\begin{frame}{Acknowledgements}
    \begin{enumerate}
        \item Dr. Alex Dugas for introducing presenter to quivers and dimer models
        
        \vfill

        \item University of the Pacific Summer Undergraduate Research Fellowships program for funding
    \end{enumerate}
\end{frame}

\begin{frame}[allowframebreaks]{References}
    \printbibliography
\end{frame}


\begin{frame}{Thanks for Listening!}
    The slides can be found on my GitHub:
    \begin{center}
        \includegraphics[scale=1]{figures/github_qr.png}
        
        \url{https://github.com/ItzSomebody/jmm2024-slides}
    \end{center}
\end{frame}