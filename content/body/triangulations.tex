\section{Triangulations of $n$-gon}

\begin{frame}{Triangulations of $n$-gon}
    In this talk, a \textbf{triangulation} of an $n$-gon is a division of the $n$-gon into triangles by some choice of non-intersecting diagonals along with the sides of the $n$-gon.
    \begin{center}
		\begin{tikzpicture}[scale=0.3,baseline=(current bounding box.center)]
			% Vertices
			\begin{scope}[every node/.style={circle,fill,draw,inner sep=2pt}]
				\node (A) at (5,0) {};
				\node (B) at (2.5,4.33) {};
				\node (C) at (-2.5,4.33) {};
				\node (D) at (-5,0) {};
				\node (E) at (-2.5,-4.33) {};
				\node (F) at (2.5,-4.33) {};
			\end{scope}
			
			% Edges
			\begin{scope}[>={Stealth[black]},
				every node/.style={fill=white,circle},
				every edge/.style={draw=black,thick}]
				
				% sides
				\path (A) edge (B);
				\path (B) edge (C);
				\path (C) edge (D);
				\path (D) edge (E);
				\path (E) edge (F);
				\path (F) edge (A);
			\end{scope}

			% Diagonals
			\begin{scope}[>={Stealth[black]},
				every node/.style={fill=white,circle,inner sep=1pt},
				every edge/.style={draw=blue,thick}]
				
				\path (A) edge (D) edge (E);
				\path (A) edge node {$a$} (C);
			\end{scope}
		\end{tikzpicture}
	\end{center}
    These are special cases of \textbf{ideal triangulations} as described by Fomin, Shapiro, and Thurston in \cite{fominClusterAlgebrasTriangulated2006}.
\end{frame}

\begin{frame}{Flipping of a Triangulation}
    We can \textbf{flip} a diagonal of the triangulation by removing a diagonal and replacing it with the other diagonal of the resulting quadrilateral.
    \begin{center}
        \begin{tikzpicture}[scale=0.3,baseline=(current bounding box.center)]
			% Vertices
			\begin{scope}[every node/.style={circle,fill,draw,inner sep=2pt}]
				\node (A) at (5,0) {};
				\node (B) at (2.5,4.33) {};
				\node (C) at (-2.5,4.33) {};
				\node (D) at (-5,0) {};
				\node (E) at (-2.5,-4.33) {};
				\node (F) at (2.5,-4.33) {};
			\end{scope}
			
			% Edges
			\begin{scope}[>={Stealth[black]},
				every node/.style={fill=white,circle},
				every edge/.style={draw=black,thick}]
				
				% sides
				\path (A) edge (B);
				\path (B) edge (C);
				\path (C) edge (D);
				\path (D) edge (E);
				\path (E) edge (F);
				\path (F) edge (A);
			\end{scope}

			% Diagonals
			\begin{scope}[>={Stealth[black]},
				every node/.style={fill=white,circle,inner sep=1pt},
				every edge/.style={draw=blue,thick}]
				
				\path (A) edge (D) edge (E);
				\path (A) edge node {$a$} (C);
			\end{scope}
		\end{tikzpicture}
		\begin{tikzpicture}[baseline=(current bounding box.center)]
			\begin{scope}[>={Stealth[black,inset=1pt,length=10pt]},
				every edge/.style={draw=black,thick}]
				
				\node (A) at (0,0) {};
				\node (B) at (1.5,0) {};
				\path[<->] (A) edge (B);
			\end{scope}
		\end{tikzpicture}
		\begin{tikzpicture}[scale=0.3,baseline=(current bounding box.center)]
			% Vertices
			\begin{scope}[every node/.style={circle,fill,draw,inner sep=2pt}]
				\node (A) at (5,0) {};
				\node (B) at (2.5,4.33) {};
				\node (C) at (-2.5,4.33) {};
				\node (D) at (-5,0) {};
				\node (E) at (-2.5,-4.33) {};
				\node (F) at (2.5,-4.33) {};
			\end{scope}
			
			% Edges
			\begin{scope}[>={Stealth[black]},
				every node/.style={fill=white,circle},
				every edge/.style={draw=black,thick}]
				
				% sides
				\path (A) edge (B);
				\path (B) edge (C);
				\path (C) edge (D);
				\path (D) edge (E);
				\path (E) edge (F);
				\path (F) edge (A);
			\end{scope}

			% Edges
			\begin{scope}[>={Stealth[black]},
				every node/.style={fill=white,circle,inner sep=1pt},
				every edge/.style={draw=blue,thick}]
				
				% diagonals
				\path (A) edge (D) edge (E);
				\path (D) edge node {$b$} (B);
			\end{scope}
		\end{tikzpicture}
    \end{center}
    In the above, we flip the triangulation on the left by exchanging the diagonal $a$ for the diagonal $b$.
\end{frame}

\begin{frame}{Quick Properties of Flips}
	\begin{enumerate}
		\item Flips are involutive (self-inverse):
		\begin{center}
			\begin{tikzpicture}[scale=0.2,baseline=(current bounding box.center)]
				% Vertices
				\begin{scope}[every node/.style={circle,fill,draw,inner sep=1.5pt}]
					\node (A) at (5,0) {};
					\node (B) at (2.5,4.33) {};
					\node (C) at (-2.5,4.33) {};
					\node (D) at (-5,0) {};
					\node (E) at (-2.5,-4.33) {};
					\node (F) at (2.5,-4.33) {};
				\end{scope}
				
				% Edges
				\begin{scope}[>={Stealth[black]},
					every node/.style={fill=white,circle},
					every edge/.style={draw=black,thick}]
					
					% sides
					\path (A) edge (B);
					\path (B) edge (C);
					\path (C) edge (D);
					\path (D) edge (E);
					\path (E) edge (F);
					\path (F) edge (A);
				\end{scope}
	
				% Diagonals
				\begin{scope}[>={Stealth[black]},
					every node/.style={fill=white,circle,inner sep=1pt},
					every edge/.style={draw=blue,thick}]
					
					\path (A) edge (D) edge (E);
					\path (A) edge node {$a$} (C);
				\end{scope}
			\end{tikzpicture}
			\begin{tikzpicture}[baseline=(current bounding box.center)]
				\begin{scope}[>={Stealth[black,inset=1pt,length=10pt]},
					every edge/.style={draw=black,thick}]
					
					\node (A) at (0,0) {};
					\node (B) at (1,0) {};
					\path[->] (A) edge (B);
				\end{scope}
			\end{tikzpicture}
			\begin{tikzpicture}[scale=0.2,baseline=(current bounding box.center)]
				% Vertices
				\begin{scope}[every node/.style={circle,fill,draw,inner sep=1.5pt}]
					\node (A) at (5,0) {};
					\node (B) at (2.5,4.33) {};
					\node (C) at (-2.5,4.33) {};
					\node (D) at (-5,0) {};
					\node (E) at (-2.5,-4.33) {};
					\node (F) at (2.5,-4.33) {};
				\end{scope}
				
				% Edges
				\begin{scope}[>={Stealth[black]},
					every node/.style={fill=white,circle},
					every edge/.style={draw=black,thick}]
					
					% sides
					\path (A) edge (B);
					\path (B) edge (C);
					\path (C) edge (D);
					\path (D) edge (E);
					\path (E) edge (F);
					\path (F) edge (A);
				\end{scope}
	
				% Edges
				\begin{scope}[>={Stealth[black]},
					every node/.style={fill=white,circle,inner sep=1pt},
					every edge/.style={draw=blue,thick}]
					
					% diagonals
					\path (A) edge (D) edge (E);
					\path (D) edge node {$b$} (B);
				\end{scope}
			\end{tikzpicture}
			\begin{tikzpicture}[baseline=(current bounding box.center)]
				\begin{scope}[>={Stealth[black,inset=1pt,length=10pt]},
					every edge/.style={draw=black,thick}]
					
					\node (A) at (0,0) {};
					\node (B) at (1,0) {};
					\path[->] (A) edge (B);
				\end{scope}
			\end{tikzpicture}
			\begin{tikzpicture}[scale=0.2,baseline=(current bounding box.center)]
				% Vertices
				\begin{scope}[every node/.style={circle,fill,draw,inner sep=1.5pt}]
					\node (A) at (5,0) {};
					\node (B) at (2.5,4.33) {};
					\node (C) at (-2.5,4.33) {};
					\node (D) at (-5,0) {};
					\node (E) at (-2.5,-4.33) {};
					\node (F) at (2.5,-4.33) {};
				\end{scope}
				
				% Edges
				\begin{scope}[>={Stealth[black]},
					every node/.style={fill=white,circle},
					every edge/.style={draw=black,thick}]
					
					% sides
					\path (A) edge (B);
					\path (B) edge (C);
					\path (C) edge (D);
					\path (D) edge (E);
					\path (E) edge (F);
					\path (F) edge (A);
				\end{scope}
	
				% Diagonals
				\begin{scope}[>={Stealth[black]},
					every node/.style={fill=white,circle,inner sep=1pt},
					every edge/.style={draw=blue,thick}]
					
					\path (A) edge (D) edge (E);
					\path (A) edge node {$a$} (C);
				\end{scope}
			\end{tikzpicture}
		\end{center}

		\vfill

		\item Number of edges invariant under flips
	\end{enumerate}
\end{frame}